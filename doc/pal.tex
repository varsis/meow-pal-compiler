\documentclass{article}
\usepackage[letterpaper, left=2.5cm, right=2.5cm, top=2.5cm, bottom=2.5cm]{geometry}
\usepackage{amsmath}
\begin{document}

\title{PAL Compiler Documentation }
\date{\today}
\author{Chris Pavlicek, Connor Moreside, Mike Armstrong, and Steve Jahns}
\maketitle

% From checkpoint 1 spec:
%
% Documentation
%
% Your documentation should describe what your group has done on the
% project. This does not mean that you should reiterate class notes; I want to
% know what is different about your group. Some things you may want to discuss
% include: handling of lexical units, syntax error reporting strategies, syntax
% error recovery (if any) , problems encountered and their solutions, etc.
% When writing, remember who is going to read it—skip the preli minaries
% and get to the point. There is a page limit of 12 double-spaced pages,
% not in cluding diagrams.

\section*{Introduction}
	% Overall design
	% Outline of classes + dependencies

\section*{Lexical Analysis}
	% handling of lexical units
	% syntax error handling at lexical level

\section*{Syntax Analysis}
	% handling of lexical units
	% syntax error reporting
	% recovery strategies

\section*{Testing Strategies}
	% gtest, unit tests
	% whole file tests

\end{document}
